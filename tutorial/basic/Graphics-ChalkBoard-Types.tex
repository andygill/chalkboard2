\subsection{Graphics.ChalkBoard.Types}     

         

 

 
\begin{tabular}{p{0.95\linewidth}}{\bf {\bf Contents}}\\ % (8)

 
\begin{itemize}
\setlength{\itemsep}{0in}

\item Basic types
\item Overlaying
\item Scaling
\item Linear Interpolation
\item Averaging
\item Constants
\item Colors
\end{itemize}
\\ % (72)


\end{tabular}


 

\subsubsection{Description}

\hspace{0.05\textwidth}\begin{minipage}{0.9\textwidth}This module contains the types used by chalkboard, except Board itself.\end{minipage}

 

\subsubsection{Synopsis}

 

 
\begin{tabular}{p{0.95\linewidth}}{type UI = R}\\ % (11)


{type R = Float}\\ % (14)


{type Point = (R, R)}\\ % (19)


{type Radian = Float}\\ % (19)


{class  Over c  where}\\ % (20)

 
\begin{tabular}{p{0.95\linewidth}}{over :: c -{\tt >} c -{\tt >} c}\\ % (31)


\end{tabular}
\\ % (31)


{stack :: Over c ={\tt >} [c] -{\tt >} c}\\ % (39)


{class  Scale c  where}\\ % (21)

 
\begin{tabular}{p{0.95\linewidth}}{scale :: R -{\tt >} c -{\tt >} c}\\ % (32)


\end{tabular}
\\ % (32)


{class  Lerp a  where}\\ % (20)

 
\begin{tabular}{p{0.95\linewidth}}{lerp :: UI -{\tt >} a -{\tt >} a -{\tt >} a}\\ % (43)


\end{tabular}
\\ % (43)


{class  Average a  where}\\ % (23)

 
\begin{tabular}{p{0.95\linewidth}}{average :: [a] -{\tt >} a}\\ % (25)


\end{tabular}
\\ % (25)


{nearZero :: R}\\ % (13)


{type Gray = UI}\\ % (14)


{data  RGB  = RGB !UI !UI !UI}\\ % (28)


{data  RGBA  = RGBA !UI !UI !UI !UI}\\ % (34)


\end{tabular}


 

 

\subsubsection{Basic types}

 

{type {\bf UI} = R}

\hspace{0.05\textwidth}\begin{minipage}{0.9\textwidth}Unit Interval: value between 0 and 1, inclusive.\end{minipage}

 

{type {\bf R} = Float}

\hspace{0.05\textwidth}\begin{minipage}{0.9\textwidth}A real number.\end{minipage}

 

{type {\bf Point} = (R, R)}

\hspace{0.05\textwidth}\begin{minipage}{0.9\textwidth}A point in R2.\end{minipage}

 

{type {\bf Radian} = Float}

\hspace{0.05\textwidth}\begin{minipage}{0.9\textwidth}Angle units\end{minipage}

 

\subsubsection{Overlaying}

 

{class  {\bf Over} c  where}

 
\begin{tabular}{p{0.95\linewidth}} For placing a value literally {\em over} another value. The 2nd value {\em might} shine through. The operation {\em must} be associative.\\ % (119)


{\bf Methods}\\ % (7)

 
\begin{tabular}{p{0.95\linewidth}}{{\bf over} :: c -{\tt >} c -{\tt >} c}\\ % (31)


\end{tabular}
\\ % (31)


{\bf  Instances}\\ % (10)

 
\begin{tabular}{p{0.95\linewidth}}{Over Bool}\\ % (9)

{Over RGBA}\\ % (9)

{Over RGB}\\ % (8)

{Over Gray}\\ % (9)

{Over (Maybe a)}\\ % (14)

{Over a ={\tt >} Over (Board a)}\\ % (30)


\end{tabular}
\\ % (30)


\end{tabular}


 

{{\bf stack} :: Over c ={\tt >} [c] -{\tt >} c}

\hspace{0.05\textwidth}\begin{minipage}{0.9\textwidth}{\tt stack} stacks a list of things over each other,  where earlier elements are {\tt over} later elements. Requires non empty lists, which can be satisfied by using an explicitly
 transparent {\tt Board} as one of the elements.\end{minipage}

 

\subsubsection{Scaling}

 

{class  {\bf Scale} c  where}

 
\begin{tabular}{p{0.95\linewidth}} {\tt Scale} something by a value. scaling value can be bigger than 1.\\ % (63)


{\bf Methods}\\ % (7)

 
\begin{tabular}{p{0.95\linewidth}}{{\bf scale} :: R -{\tt >} c -{\tt >} c}\\ % (32)


\end{tabular}
\\ % (32)


{\bf  Instances}\\ % (10)

 
\begin{tabular}{p{0.95\linewidth}}{Scale RGB}\\ % (9)

{Scale R}\\ % (7)

{Scale (Board a)}\\ % (15)

{(Scale a, Scale b) ={\tt >} Scale ((,) a b)}\\ % (43)


\end{tabular}
\\ % (43)


\end{tabular}


 

\subsubsection{Linear Interpolation}

 

{class  {\bf Lerp} a  where}

 
\begin{tabular}{p{0.95\linewidth}} Linear interpolation between two values.\\ % (40)


{\bf Methods}\\ % (7)

 
\begin{tabular}{p{0.95\linewidth}}{{\bf lerp} :: UI -{\tt >} a -{\tt >} a -{\tt >} a}\\ % (43)


\end{tabular}
\\ % (43)


{\bf  Instances}\\ % (10)

 
\begin{tabular}{p{0.95\linewidth}}{Lerp RGB}\\ % (8)

{Lerp R}\\ % (6)

{Lerp a ={\tt >} Lerp (Maybe a)}\\ % (30)

{(Lerp a, Lerp b) ={\tt >} Lerp ((,) a b)}\\ % (40)


\end{tabular}
\\ % (40)


\end{tabular}


 

\subsubsection{Averaging}

 

{class  {\bf Average} a  where}

 
\begin{tabular}{p{0.95\linewidth}} {\tt Average} a set of values. weighting can be achived using multiple entries.\\ % (73)


{\bf Methods}\\ % (7)

 
\begin{tabular}{p{0.95\linewidth}}{{\bf average} :: [a] -{\tt >} a}\\ % (25)

\hspace{0.05\textwidth}\begin{minipage}{0.9\textwidth}average is not defined for empty list\end{minipage}\\ % (37)


\end{tabular}
\\ % (37)


{\bf  Instances}\\ % (10)

 
\begin{tabular}{p{0.95\linewidth}}{Average RGB}\\ % (11)

{Average R}\\ % (9)

{(Average a, Average b) ={\tt >} Average ((,) a b)}\\ % (49)


\end{tabular}
\\ % (49)


\end{tabular}


 

\subsubsection{Constants}

 

{{\bf nearZero} :: R}

\hspace{0.05\textwidth}\begin{minipage}{0.9\textwidth}Close to zero; needed for {\tt Over (Alpha c)} instance.\end{minipage}

 

\subsubsection{Colors}

 

{type {\bf Gray} = UI}

\hspace{0.05\textwidth}\begin{minipage}{0.9\textwidth}{\tt Gray} is just a value between 0 and 1, inclusive. Be careful to consider if this is pre or post gamma.
\end{minipage}

 

{data  {\bf RGB}  }

 
\begin{tabular}{p{0.95\linewidth}} {\tt RGB} is our color, with values between 0 and 1, inclusive.\\ % (57)

{\bf Constructors}\\ % (12)

 
\begin{tabular}{p{0.95\linewidth}} {\bf RGB} !UI !UI !UI \\ % (15)


\end{tabular}
\\ % (15)

{\bf  Instances}\\ % (10)

 
\begin{tabular}{p{0.95\linewidth}}{Show RGB}\\ % (8)

{Binary RGB}\\ % (10)

{Average RGB}\\ % (11)

{Lerp RGB}\\ % (8)

{Scale RGB}\\ % (9)

{Over RGB}\\ % (8)

{Obs RGB}\\ % (7)


\end{tabular}
\\ % (7)


\end{tabular}


 

{data  {\bf RGBA}  }

 
\begin{tabular}{p{0.95\linewidth}} {\tt RGBA} is our color, with values between 0 and 1, inclusive. These values are *not* prenormalized
\\ % (96)

{\bf Constructors}\\ % (12)

 
\begin{tabular}{p{0.95\linewidth}} {\bf RGBA} !UI !UI !UI !UI \\ % (20)


\end{tabular}
\\ % (20)

{\bf  Instances}\\ % (10)

 
\begin{tabular}{p{0.95\linewidth}}{Show RGBA}\\ % (9)

{Binary RGBA}\\ % (11)

{Over RGBA}\\ % (9)

{Obs RGBA}\\ % (8)


\end{tabular}
\\ % (8)


\end{tabular}


 

 

