\subsection{Graphics.ChalkBoard.Board}     

  

 

 
\begin{tabular}{p{0.95\linewidth}}{\bf {\bf Contents}}\\ % (8)

 
\begin{itemize}
\setlength{\itemsep}{0in}

\item The {\tt Board} 
\item Ways of manipulating {\tt Board}.
\item Ways of creating a new {\tt Board}.
\end{itemize}
\\ % (66)


\end{tabular}


 

 

\subsubsection{Synopsis}

 

 
\begin{tabular}{p{0.95\linewidth}}{data  Board a}\\ % (13)


{({\tt <}\${\tt >}) ::  (O a -{\tt >} O b) -{\tt >} Board a -{\tt >} Board b}\\ % (75)


{move ::  (R, R) -{\tt >} Board a -{\tt >} Board a}\\ % (49)


{rotate ::  Radian -{\tt >} Board a -{\tt >} Board a}\\ % (51)


{scaleXY ::  (R, R) -{\tt >} Board a -{\tt >} Board a}\\ % (52)


{boardOf ::  O a -{\tt >} Board a}\\ % (32)


{circle :: Board Bool}\\ % (20)


{box :: (Point, Point) -{\tt >} Board Bool}\\ % (41)


{square :: Board Bool}\\ % (20)


{triangle :: Point -{\tt >} Point -{\tt >} Point -{\tt >} Board Bool}\\ % (67)


{polygon :: [Point] -{\tt >} Board Bool}\\ % (38)


{readBoard :: String -{\tt >} IO (Int, Int, Board RGBA)}\\ % (54)


{readNormalizedBoard :: String -{\tt >} IO (Int, Int, Board RGBA)}\\ % (64)


\end{tabular}


 

 

\subsubsection{The {\tt Board} }

 

{data  {\bf Board} a }

 
\begin{tabular}{p{0.95\linewidth}}{\bf  Instances}\\ % (10)

 
\begin{tabular}{p{0.95\linewidth}}{Show (Board a)}\\ % (14)

{Scale (Board a)}\\ % (15)

{Over a ={\tt >} Over (Board a)}\\ % (30)


\end{tabular}
\\ % (30)


\end{tabular}


 

\subsubsection{Ways of manipulating {\tt Board}.}

 

{{\bf ({\tt <}\${\tt >})} ::  (O a -{\tt >} O b) -{\tt >} Board a -{\tt >} Board b}

\hspace{0.05\textwidth}\begin{minipage}{0.9\textwidth}{\tt fmap} like operator over a {\tt Board}.\end{minipage}

 

{{\bf move} ::  (R, R) -{\tt >} Board a -{\tt >} Board a}

\hspace{0.05\textwidth}\begin{minipage}{0.9\textwidth}{\tt move} moves the contents of {\tt Board}\end{minipage}

 

{{\bf rotate} ::  Radian -{\tt >} Board a -{\tt >} Board a}

\hspace{0.05\textwidth}\begin{minipage}{0.9\textwidth}{\tt rotate} rotates a {\tt Board} clockwise by a radian argument.\end{minipage}

 

{{\bf scaleXY} ::  (R, R) -{\tt >} Board a -{\tt >} Board a}

\hspace{0.05\textwidth}\begin{minipage}{0.9\textwidth}{\tt scaleXY} scales the contents of {\tt Board} the X and Y dimension.  See also {\tt scale}.\end{minipage}

 

\subsubsection{Ways of creating a new {\tt Board}.}

 

{{\bf boardOf} ::  O a -{\tt >} Board a}

\hspace{0.05\textwidth}\begin{minipage}{0.9\textwidth}pure like operator for {\tt Board}.	\end{minipage}

 

{{\bf circle} :: Board Bool}

\hspace{0.05\textwidth}\begin{minipage}{0.9\textwidth}Generate a unit circle (radius .5) centered on origin\end{minipage}

 

{{\bf box} :: (Point, Point) -{\tt >} Board Bool}

\hspace{0.05\textwidth}\begin{minipage}{0.9\textwidth}{\tt box} generate a box between two corner points)\end{minipage}

 

{{\bf square} :: Board Bool}

\hspace{0.05\textwidth}\begin{minipage}{0.9\textwidth}Generate a unit square (1 by 1 square) centered on origin\end{minipage}

 

{{\bf triangle} :: Point -{\tt >} Point -{\tt >} Point -{\tt >} Board Bool}

\hspace{0.05\textwidth}\begin{minipage}{0.9\textwidth}Generate an arbitary triangle from 3 points.\end{minipage}

 

{{\bf polygon} :: [Point] -{\tt >} Board Bool}

\hspace{0.05\textwidth}\begin{minipage}{0.9\textwidth}Generate a (convex) polygon from a list of points. There must be at least 3 points, and the points must form a convex polygon.
\end{minipage}

 

{{\bf readBoard} :: String -{\tt >} IO (Int, Int, Board RGBA)}

\hspace{0.05\textwidth}\begin{minipage}{0.9\textwidth}read a file containing a common image format (jpg, gif, etc.), and create a 'Board RGBA', and the X and Y size of the image.\end{minipage}

 

{{\bf readNormalizedBoard} :: String -{\tt >} IO (Int, Int, Board RGBA)}

 

 

